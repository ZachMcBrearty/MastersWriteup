\documentclass[12pt, onecolumn]{revtex4-2}    % Font size (12pt) and column number (one or two).

\usepackage{times}                          % Times New Roman font type

\usepackage[a4paper, left=2.5cm, right=2.5cm,
 top=2.5cm, bottom=2.5cm]{geometry}       % Defines paper size and margin length

\renewcommand{\baselinestretch}{1.15}     % Defines the line spacing
% \setlength{\parindent}{0pt}
\errorcontextlines=20

\usepackage[font=small, labelfont=bf]{caption} % Defines caption font size and caption title bolded

\usepackage{graphics,graphicx,epsfig,ulem}	% Makes sure all graphics works
\usepackage{amsmath} 						% Adds mathematical features for equations

\usepackage{etoolbox}                       % Customise date to preferred format
\makeatletter
\patchcmd{\frontmatter@RRAP@format}{(}{}{}{}
\patchcmd{\frontmatter@RRAP@format}{)}{}{}{}
\renewcommand\Dated@name{}
\makeatother

\usepackage{fancyhdr}

\usepackage{multirow}

% \usepackage[ddmmyyyy]{datetime2}

\pagestyle{fancy}                           % Insert header
\renewcommand{\headrulewidth}{0pt}
\lhead{\small Zachery McBrearty}                          % Your name
\rhead{\small Orbital Constraints on Exoplanet Habitability}            % Your report title               

\def\thesection{\arabic{section}}

\def\bibsection{\section*{References}}        % Position reference section correctly

\newcommand{\heatcap}{\ensuremath{\ \text{Jm}^{-2} \text{K}^{-1}}}
\newcommand{\diffusion}{\ensuremath{\ \text{Wm}^{-2} \text{K}^{-1}}}

\newcommand{\radians}{\ensuremath{^{\text{rad}}}}
\newcommand{\degrees}{\ensuremath{^{\circ}}}

\newcommand{\partialderiv}[2]{\frac{\partial {#1}}{\partial {#2}}}
\newcommand{\partialderivsecnd}[2]{\frac{\partial^2 {#1}}{\partial {#2}^2}}

%%%%% Document %%%%%
\begin{document}


\title{Orbital Constraints on Exoplanet Habitability}
\date{Submitted: \today{}}
\author{Zachery McBrearty}
\affiliation{\normalfont Level 4 Project, MPhys Physics with Astronomy\\ Supervisor: Dr Richard Wilman \\ Second Supervisor: Dr Craig Testrow \\ Department of Physics, Durham University}

\begin{abstract}

    A 1-D energy balance climate model is developed in order to investigate how changing certain orbital parameters can result in changes to a planet's habitability.
    Theoretical relationships between temperature, semimajoraxis, and eccentricity are derived from a simple 0-D energy balance model and are tested against the 1-D model and are found to be correct.
    A qualitative analysis of obliquity shows that there are optimal obliquities to minimise and maximise global temperature.
    The climates of exomoons orbiting gas giants are also investigated, including reflected light from the gas giant, eclipsing, and tidal heating.
    It is expected that these additional sources of heat move the habitable zones for the planet outwards.

\end{abstract}


\maketitle
%\thispagestyle{plain} % produces page number for front page

\tableofcontents
% \let\toc@pre\relax
% \let\toc@post\relax

\newpage

\section{Introduction} \label{sec:Introduction}
% Exoplanet/Exomoon habitability motivation 
% > Search for extraterrestrial life
% > Settling other worlds
% > Current missions to find exoplanets and exomoons


% Climate model
% > 0D-Equation and where it comes from
\begin{equation}
    \pi r^2 S(1-A) = 4 \pi r^2 \sigma T^4, \label{eq:0DEBCM}
\end{equation}
% > 1D-Equation + what the parameters are 
\begin{equation}
    C \partialderiv{T(t, x)}{t} - D \partialderiv{}{x} \left((1-x^2)\partialderiv{T(x, t)}{x}\right) + I - S(1-A) = 0
    \label{eq:PDE_in_x}
\end{equation}
% > Ignoring longitude dimension (How/Why)
% Earth as a baseline
% > using Earth to constrain certain parameters of the model to be realistic
% Exoplanets (i.e. moving Earth around)
% > Describe how varying parameters can lead to habitability parameter space
% > Suggest by generating parameter spaces that new discoveries can be quickly checked  
% Exomoons
% > How they differ from exoplanets
% > TH, Eclipsing, Reflectance
% > Depth model -> suggestion of life without requirements of light under an ice shell

\section{1-D Energy Balance Climate Model}\label{sec:1DEBCM}
% Rederive 1DEBCM from standard heat eqn
The EBCM can be derived from the standard heat equation given by
\begin{equation}
    \partialderiv{T}{t} = \alpha \nabla^2 T,
\end{equation}
where $T(t, r, \theta, \phi)$ is the temperature at time $t$, radius $r$, co-latitude $\theta$, and longitude $\phi$.
The constant $\alpha$ is related to the heat capacity and diffusion rate of the system.
Expanding the laplacian in spherical coordinates the equation becomes
\begin{equation}
    \partialderiv{T}{t} = \alpha \left[\frac{1}{r} \partialderivsecnd{}{r} (r T)
        + \frac{1}{r^2 \sin\theta} \partialderiv{}{\theta}\left(\sin\theta \partialderiv{T}{\theta}\right)
        + \frac{1}{r^2 \sin^2\theta} \partialderivsecnd{T}{\theta} \right]. \label{eq:FullyExpandedHeatEqn}
\end{equation}
The EBCM is arrived at by first letting $T(r, \theta, \phi) = T(\lambda)$, with $\lambda = \pi - \theta$. Thus the equation simplifies to
\begin{align}
    \partialderiv{T}{t} & = \frac{\alpha}{r^2 \sin\theta} \partialderiv{}{\theta}\left(\sin\theta \partialderiv{T}{\theta}\right)   \\
                        & = \frac{\alpha}{r^2} \left(\tan\lambda \partialderiv{T}{\lambda} + \partialderivsecnd{T}{\lambda}\right).
\end{align}
The original equation can be recovered by defining $\alpha / r^2 \equiv D / C$ for diffusion constant $D$ and heat capacity $C$.
Then adding incoming solar radiation $S$ (insolation), which is reduced by planetary albedo $A$, and outgoing IR-emission $I$ to the PDE.
Thus the form of the EBCM in terms of latitude (instead of $x = \sin\lambda$) is given by
\begin{equation}
    C\partialderiv{T}{t} = D \left(\tan\lambda \partialderiv{T}{\lambda} + \partialderivsecnd{T}{\lambda}\right) + S(1-A) - I.
\end{equation}
% Expand derivatives and introduce latitude as the standard parameter
\subsection{Characterising Model Parameters} \label{ssec:CharacterisingModelParameters} %%% -> could appendix if not enough space %%%
In this analysis the forms of the model parameters are taken from Williams and Kastings (WK97) \cite{WK97}.

% Diffusion
The diffusion varies with orbital and atmospheric parameters as,
\begin{equation}
    \frac{D}{D_0} = \frac{p}{p_0} \frac{c_p}{c_{p,0}} \left(\frac{m}{28}\right)^{-2} \left(\frac{\Omega}{1 \text{day}^{-1}}\right)^{-2}
\end{equation}
where $D_0 = 0.56 \diffusion$ is from fitting to the time averaged Earth model from North and Coakley 1979 \cite{NC79}.
$p$ is the atmospheric pressure relative to $p_0 = 101 \ \text{kPa}$.
$c_p$ is the heat capacity of the atmosphere, relative to $c_{p,0} = 1\times10^3 \ \text{g}^{-1} \text{K}^{-1}$.
$m$ is the (average) mass of the particles in the atmosphere, relative to the Nitrogen molecule.
$\Omega$ is the rotation rate of the planet, relative to Earth's $1$ rotation per day.
This can be extended to be time variable, such as having CO$_2$ emissions increase pressure, change heat capacity, and increase mass of particles.
However this paper only considers varying the rotation rate of the planet.

% Heat capacity
Heat capacity, $C(\lambda, t)$, varies with latitude through the ocean-land fraction, $f_o(\lambda)$, and with temperature through the ice-ocean fraction, $f_i(T)$, as
\begin{equation}
    C(\lambda, T) = (1 - f_o(\lambda)) C_{land} + f_o(\lambda) ((1-f_i(T)) C_{ocean} + f_i(T) C_{ice}(T)),
\end{equation}
Where $C_{\text{land}} = 5.25\times10^6 \heatcap$ and $C_{\text{ocean}} = 40 \times C_{\text{land}}$ are constant, and
\begin{equation}
    C_{\text{ice}}(T) =
    \begin{cases}
        9.2 C_\text{land} & T \ge 263\text{K} \\
        2.0 C_\text{land} & T < 263\text{K},
    \end{cases}
\end{equation}

% IR-emission
% Albedo -> snow / water reflectance
WK97 provides three sets of IR-emission and Albedo functions. Following the example of SMS08 and Dressing et al 2010 (here on Dressing10) \cite{Dressing10} the second set of IR and Albedo functions which are given by
\begin{equation}
    I(T) = I_2(T) = \frac{\sigma T^4}{1 + 0.5925 (T / 273 \text{K}) ^ 3}
\end{equation}
\begin{equation}
    A(T) = A_2(T) = 0.525 - 0.245 \tanh\left(\frac{T - 268 \text{K}}{5}\right),
\end{equation}
are used in all models.
This IR-emission is a blackbody radiation term (numerator) damped by the optical thickness of the atmosphere (denominator) which is roughly equivalent to a greenhouse gas effect.
The albedo function is a smooth scaling from low to high reflectivity due to snow and water-vapour reflectance.

% Insolation -> why diurnally averaged (No longitude)
The insolation function, $S$, is defined in WK97 as the day averaged incident (based on latitude) radiation from the sun,
$$
    S(\lambda, t) = \frac{q_0}{\pi} \left(\frac{1 \ \text{au}}{a}\right)^2 (H(t) \sin{\lambda} \sin{\delta(t)} + \cos{\lambda} \cos{\delta(t)} \sin{H(t)})
$$
where $q_0=1360 \ \text{Wm}^{-2}$ is the insolation from the Sun,
$a$ is the distance from the Sun,
$\cos H(t) = -\tan \lambda \tan \delta(t)$ is the radian half-day length with $0 < H < \pi$,
and $\delta(t)$ is the solar declination defined by
$$
    \sin \delta(t) = -\sin \delta_0 \cos(L_s(t) + \pi/2)
$$
where $\delta_0$ is the obliquity of the planet and $L_s(t) = \omega t$ is orbital longitude from an orbital angular velocity found by Kepler's laws.

\subsection{Discretisation of the climate model} \label{ssec:DiscretisationPDE}
% choice of discretisation
Numerically integrating the EBCM requires the derivatives to be discretised.
Spatially the planet can be split into $S$ nodes, separated by
\begin{equation}
    \Delta\lambda = \frac{\pi\radians}{S-1} = \frac{180\degrees}{S-1},
\end{equation}
with spatial indexing of each zone from $m=0, 1, \dots, S-1$.
Similarly, a temporal indexing of $n=0, 1, \dots$ is used to discretise time in steps of $\Delta t$.

% choice of partial derivatives
The spatial derivatives can then be approximated by the central difference and second order central difference:
\begin{align}
    \partialderiv{T^m_n}{\lambda}      & = \frac{T^{m+1}_n - T^{m-1}_n}{2 \Delta\lambda},     \label{eq:space_1}        \\
    \partialderivsecnd{T^m_n}{\lambda} & = \frac{T^{m+2}_n -2T^m_n + T^{m-2}_n}{(2 \Delta\lambda)^2},\label{eq:space_2}
\end{align}
and the temporal derivative can be approximated as a forward difference,
\begin{equation} \label{eq:time_1}
    \partialderiv{T^m_n}{t} = \frac{T^m_{n+1} - T^m_n}{\Delta t},
\end{equation}
with numerical stability analysed in appendix \ref{appx:NumStability}.
Evolving the EBCM is performed by solving eqn. \ref{eq:time_1} for $T^m_{n+1}$ in terms of the parameter and temperature values at time $n$.

% choice of edge case partial derivatives
However, a problem arises at the edges of the model as $m=-2, -1, S, S+1$ are not defined.
To fix this the derivatives at $m=0$ ($m=S-1$) are discretised as forward then backward (backward then forward) derivatives.
By imposing that ${\partial T^{m=0, S-1}_n}/{\partial \lambda} = 0$, these second order derivatives reduce to
\begin{alignat}{2}
    \partialderivsecnd{T^{m=0}_n}{\lambda}   & = \left(\partialderiv{T^{m=1}_n}{\lambda} - \partialderiv{T^{m=0}_n}{\lambda}\right) / \Delta\lambda     &  & = \frac{T^{m=1}_n - T^{m=0}_n}{(\Delta\lambda)^2}
    \label{eq:forward_backward}                                                                                                                                                                                     \\
    \partialderivsecnd{T^{m=S-1}_n}{\lambda} & = \left(\partialderiv{T^{m=S-1}_n}{\lambda} - \partialderiv{T^{m=S-2}_n}{\lambda}\right) / \Delta\lambda &  & = \frac{T^{m=S-2}_n - T^{m=S-1}_n}{(\Delta\lambda)^2}.
    \label{eq:backward_forward}
\end{alignat}
Furthermore, the treatment imposed for the $m=1$ and $m=S-2$ second order derivatives is much the same, using central-backward and central-forward derivatives respectively.


\section{Earth-like model} \label{sec:EarthLikeModel}
% choice of 0.7-uniform
% choice of D_0 by fitting to NC97
% choice of spatial separation and time step sizes
% > relate to numerical stability appx 
% Show Earth: 
% > temperature distribution
% > habitability (human, bio) distribution
% > Climate cycles (eccentricity, obliquity)
\begin{table*}
    \begin{tabular}{|c|c|c|c|c|c|}
        \hline
        Semimajoraxis        & Eccentricity     & Obliquity     & No. spatial nodes & Timestep         & \multirow{2}{*}{Land fraction type} \\
        $a_{\text{gas}}$, au & $e_{\text{gas}}$ & $\delta$, deg & $S$               & $\Delta t$, days &                                     \\
        \hline
        1                    & 0.0167           & 23.5          & 61                & 1                & Uniform 70\% Ocean                  \\
        \hline
    \end{tabular}
    \caption{A summary of the default parameters for the Earth-like model.
        A `Uniform' land fraction indicates that the model has the same ratio of land to ocean across the entire planet.
        The odd number of spatial nodes means there is a true equator with $\lambda = 0$ as well as poles with $\lambda = \pm 90\degrees$}
    \label{tab:default_params}
\end{table*}

\section{Exoplanets} \label{sec:Exoplanets}
\subsection{Investigating time-averaged solar flux} \label{ssec:InvTimeAveragedSolarFlux}
% use 0D model and Mendez to find parameter relations
\begin{equation}
    \langle S \rangle = \frac{L}{a^2 \sqrt{1-e^2}} \label{eq:avgInsolation}
\end{equation}
\begin{equation}
    \frac{L (1-A)}{a^2 \sqrt{1-e^2}} = \sigma T^4
\end{equation}
\begin{equation}
    T \propto a^{-1/2}
\end{equation}
\begin{equation}
    T \propto (1-e^2)^{-1/8}
\end{equation}
\begin{equation}
    a \propto (1-e^2)^{-1/4}
\end{equation}
% > T prop a^-1/2
% > > talk about powerlaw being obeyed well until snowball
% > T prop (1-e^2)^-1/8
% > > good relationship until very high eccentricity
% > a prop (1-e^2)^-1/4
% > > finding inner and outer habitable zone (Human, Bio)
% % Describe in caption, analysis in main body of text 
\subsection{Investigating obliquity and rotation rate}\label{ssec:InvObliquityRotationRate}
% Qualitative look at how obliquity influences habitability
% > global temperature-obliquity
% > latitude-obliquity
% > time-obliquity (seen in Earth-like model?)
% Qualitative look at how rotation-rate influences habitability
% > "island of stability" at higher rotation rates
% > Three plot of obliquity showing the "island" growing / stabilising
\subsection{Investigating of ocean fraction} \label{ssec:InvOceanFraction}% -> lower priority
% Stability of the snowballing w.r.t. ocean fraction
% > a-f_ocean plot showing snowball transition changing?

\section{Exomoons} \label{sec:Exomoons}
\subsection{Investigating eclipsing} \label{ssec:InvEclipsing}
% Eclipsing amount vs a_gas, e_gas, a_moon, e_moon
% show that eclipsing is minor / could be approximated by powerlaws?
\subsection{Investigating tidal heating} \label{ssec:InvTidalHeating}
% with constant TH -> vary a_gas-e_gas to see how habitability moves
% const a_gas/e_gas -> vary a_moon-e_moon to see habitability relations
% vary planet size/radius/mass? 
\subsection{Developing and investigating a depth model} \label{ssec:DevInvDepthModel} % -> lower priority
% Derive depth model, referencing sec2
% setup system for ocean planet habitability under an ice layer with tidal heating

\section{Discussion} \label{sec:Discussion}
% Problems arising during analysis / method
% Sources of uncertainty in the method
% Approximations made and their validity
% Next steps in the work
% > most likely about the depth model and where it could be taken

\section{Conclusion} \label{sec:Conclusion}
% Summary
% > Why results matter
% > What the results are
% > Main problems with results
% > Main future goals of the research


% \begin{acknowledgments}
%     (OPTIONAL) The author would like to thank...
% \end{acknowledgments}

\bibliographystyle{ieeetr}
\bibliography{references}

\clearpage

\appendix

\section{Defining Equilibrium Temperature and Averages used} \label{appx:EquilTempAverages}
% Time-averaged temperature -> T(lat)
% Latitude-averaged temperature -> T(t)
% Time-and-latitude-averaged temperature -> <T>
% Equilibrium temperature as when <T_i+1> - <T_i> / <T_i> < epsilon

\section{Numerical stability of the 1D EBCM} \label{appx:NumStability}
% Maths analysis
% Varying spacedim to check analysis
% Varying timestep to check analysis

\section{Tidal heating equations and method} \label{appx:TidalHeatingEquationsMethod}
% What the equations are
% How they are combined

\clearpage

\section*{Scientific Summary for a General Audience}

Many interesting solar systems have been reported in the news, such as the Trappist-1 system which is filled with Earth-like planets.
Simulations and models such as those in this paper are used to determine if a planet could be habitable.
A habitable zone can be made by varying the parameters of the model to see where the model is habitable, partially habitable, or uninhabitable.

The main model in this paper takes a planet and divides it into a number of latitude bands which can have energy flow between them.
Certain parameters, such as how the planet orbits around it's star and the angle the planet is tilted at, are varied to build this habitable zone.
A result of this paper is if the Earth orbited slightly further away from the Sun then it is likely that it would fall into an ice age similar to what the Earth has experienced in the past.
Another result found is that the tilt of the planet can affect how hot or cold it is, and indicates that the current tilt of the Earth gives a cold planet.

Another aspect of this paper's exoplanet research is exomoons orbiting a gas giant such as Jupiter.
In certain configurations an exomoon can be heated not only from the host star, but also due to a process called tidal heating.
Tidal heating is similar to stretching an elastic band.
Stretching and relaxing an elastic band many times can cause the band to warm up.
The moon of a gas planet is stretched slightly by unequal forces of gravity as one part of the moon is further away than the other.
If the moon's orbit is not circular then the moon is stretched and relaxed, thus heats up in a similar way to the elastic band.
Adding tidal heating to the model allows for investigations into how tidal heating can move, or change the shape of, the habitable zone.

\end{document}