\documentclass[12pt, onecolumn]{revtex4-2}    % Font size (12pt) and column number (one or two).

\usepackage{times}                          % Times New Roman font type

\usepackage[a4paper, left=2.5cm, right=2.5cm,
 top=2.5cm, bottom=2.5cm]{geometry}       % Defines paper size and margin length

\renewcommand{\baselinestretch}{1.15}     % Defines the line spacing

\usepackage[font=small,
labelfont=bf]{caption}                      % Defines caption font size and caption title bolded

\usepackage{graphics,graphicx,epsfig,ulem}	% Makes sure all graphics works
\usepackage{amsmath} 						% Adds mathematical features for equations

\usepackage{etoolbox}                       % Customise date to preferred format
\makeatletter
\patchcmd{\frontmatter@RRAP@format}{(}{}{}{}
\patchcmd{\frontmatter@RRAP@format}{)}{}{}{}
\renewcommand\Dated@name{}
\makeatother

\usepackage{fancyhdr}


\pagestyle{fancy}                           % Insert header
\renewcommand{\headrulewidth}{0pt}
\lhead{\small Zachery McBrearty}                          % Your name
\rhead{\small Orbital Constraints on Exoplanet Habitability}            % Your report title               
\def\thesection{\arabic{section}}

\def\bibsection{\section*{References}}        % Position reference section correctly


%%%%% Document %%%%%
\begin{document}


\title{Orbital Constraints on Exoplanet Habitability}
\date{Submitted: \today{}}
\author{Zachery McBrearty}
\affiliation{\normalfont Level 4 Project, MPhys Physics with Astronomy\\ Supervisor: Doctor R. Wilman \\ Second Supervisor: Dr Craig Testrow \\ Department of Physics, Durham University}

\begin{abstract}

    Implementing a 1-D energy balance climate model in order to investigate how changing certain orbital parameters can result in changes to habitability.

\end{abstract}


\maketitle
%\thispagestyle{plain} % produces page number for front page

\tableofcontents
% \let\toc@pre\relax
% \let\toc@post\relax

\newpage

\section{Introduction}

%%% example text follows %%%
The climate of a planet such as the Earth is many dimensional.

The simplest climate model is a 0 dimensional energy balance model.
The 0-D model is a simple equality of the input energy from the Sun (LHS) to the output energy from the Earth acting as a black body (RHS)
$$
    \pi r_{Earth}^2 S(1-A) = 4 \pi r_{Earth}^2 \sigma T_{Earth}^4,
$$
where $r_{Earth}$ is the radius of the Earth, $S$ is the incident solar radiation (insolation), $A$ is the reflectance (albedo) of the Earth, and $\sigma$ is the Stefan-Boltzmann constant.
This 0-D model is essentially an average over all degrees of freedom of the Earth, including rotation and orbiting of the star.

A 1-D climate model attempts to resolve the surface of the planet into latitude bands.
Since rotation of the planet would imply another dimension (namely longitude) the rotation of the planet is still averaged over.
Each of these latitude bands is treated as balancing energy in from the sun and energy out via blackbody radiation, but an additional energy diffusion term is included in the equation for energy transport between latitude bands.
$$
    C(x, t)\frac{dT(x, t)}{dt} - \frac{d}{dx} \left(D(x, t)(1-x^2)\frac{dT(x, t)}{dx}\right) + I(x, t) - S(x, t)(1-A(x, t)) = 0
$$
where $x=sin(\lambda)$, $\lambda$ is the latitude, $C$ is the heat capacity of the latitude band, $D$ is the diffusion coefficient, $I$ is the IR-emission of the band,
$S$ is the insolation, and $A$ is the albedo.

In this analysis we adopt the form of the heat capacity given by !!!.
In short: $C(x, t)$ varies with latitude through the ocean-land fraction, $f_o(x)$, and with Temperature through the ice-ocean fraction, $f_i(T)$, as
$$
    C(x, T) = (1 - f_o(x)) C_{land} + f_o(x) ((1-f_i(T)) C_{ocean} + f_i(T) C_{ice}(T)),
$$
Where $C_{land} = !!!$ and $C_{ocean} = !!!$ are constant, and
$$
    C_{ice}(T) =
    \begin{cases}
        !!! & T < 263   \\
        !!! & T >= 263,
    \end{cases}
$$

We use a diffusion coefficient which is constant in space and time, but varies with orbital and atmospheric parameters as,
$$
    \frac{D}{D_0} = \frac{p}{p_0} * \frac{c_p}{c_{p,0}} * \left(\frac{m}{28}\right)^{-2} * \left(\frac{\Omega}{1 day^{-1}}\right)^{-2}
$$
where $D_0 = 0.56$ J s$^{-1}$ m$^{-2}$ K$^{-1}$ from fitting to an Earth model (see !!!),
$p$ is the atmospheric pressure relative to $p_0 = 101$ kPa.
$c_p$ is the heat capacity of the atmosphere, relative to $c_{p,0} = 10^3$ g$^{-1}$ K$^{-1}$.
$m$ is the (average) mass of the particles in the atmosphere, relative to the Nitrogen molecule.
$\Omega$ is the rotation rate of the planet, relative to Earth's $1$ rotation per day.

IR-emission and Albedo functions are taken from !!! and are given by
$$
    I(T) = I_2(T) = \sigma T^4 / (1 + 0.5925 (T/273) ^ 3) \\
    A(T) = A_2(T) = 0.525 - 0.245 \tanh{\frac{T-268}{5}}
$$
where this IR-emission is a blackbody radiation term damped the optical thickness of the atmosphere.
and the albedo is a smooth scaling from low to high reflectivity due to snow and water-vapour reflectance.

Insolation function is defined as the day averaged incident (based on latitude) radiation from the sun,
$$
    S(\lambda, t) = \frac{q_0}{\pi} \left(\frac{1 AU}{a}\right)^2 (H \sin{\lambda} \sin{\delta} + \cos{\lambda} \cos{\delta} \sin{H})
$$
$q_0 = 1360$ W m$^{-2}$, $\lambda$ is latitude,

\section{Section heading}

Duis eget tellus tortor. Cum sociis natoque penatibus et magnis dis parturient montes, nascetur ridiculus mus. In tellus nulla, sodales eu pulvinar at, accumsan quis magna. Nunc sed lacus diam. Nam enim mauris, imperdiet ut egestas quis, tincidunt at odio. Ut viverra nulla at libero dictum aliquet. Suspendisse lacus lacus, imperdiet nec elit nec, ullamcorper facilisis ex.

\subsection{Subsection heading}

Proin sit amet mauris tincidunt, consectetur nisi ultrices, dapibus elit. Nullam vitae faucibus odio, pharetra ultrices tortor. Class aptent taciti sociosqu ad litora torquent per conubia nostra, per inceptos himenaeos.

\section{Conclusions}

Donec finibus, tellus sit amet luctus sodales, lectus ante accumsan ligula, at condimentum lorem justo a sapien. Phasellus vel tortor vitae metus lacinia efficitur ac vel ex. Aenean eget congue leo. Aliquam cursus mauris sit amet arcu dignissim, vel condimentum nisi sodales.

\begin{acknowledgments}
    (OPTIONAL) The author would like to thank...
\end{acknowledgments}

\begin{thebibliography}{}
    \bibitem{ref01} A.~N.~Other, Title of the book, edition, publishers, place of publication (year of publication), p.~123.  % example book reference 
    \bibitem{ref02} A.~N.~Other, Title of the article, journal title, volume, 123--456 (year of publication).   % example journal reference
\end{thebibliography}


\end{document}