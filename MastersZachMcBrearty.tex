\documentclass[12pt, onecolumn]{revtex4-2}    % Font size (12pt) and column number (one or two).

\usepackage{times}                          % Times New Roman font type

\usepackage[a4paper, left=2.5cm, right=2.5cm,
 top=2.5cm, bottom=2.5cm]{geometry}       % Defines paper size and margin length

\renewcommand{\baselinestretch}{1.15}     % Defines the line spacing
% \setlength{\parindent}{0pt}
\errorcontextlines=20

\usepackage[font=small, labelfont=bf]{caption} % Defines caption font size and caption title bolded

\usepackage{graphics,graphicx,epsfig,ulem}	% Makes sure all graphics works
\usepackage{amsmath} 						% Adds mathematical features for equations

\usepackage{etoolbox}                       % Customise date to preferred format
\makeatletter
\patchcmd{\frontmatter@RRAP@format}{(}{}{}{}
\patchcmd{\frontmatter@RRAP@format}{)}{}{}{}
\renewcommand\Dated@name{}
\makeatother

\usepackage{fancyhdr}

\usepackage{multirow}

% \usepackage[ddmmyyyy]{datetime2}

\pagestyle{fancy}                           % Insert header
\renewcommand{\headrulewidth}{0pt}
\lhead{\small Zachery McBrearty}                          % Your name
\rhead{\small Orbital Constraints on Exoplanet Habitability}            % Your report title               

\def\thesection{\arabic{section}}

\def\bibsection{\section*{References}}        % Position reference section correctly

\newcommand{\heatcap}{\ensuremath{\ \text{Jm}^{-2} \text{K}^{-1}}}
\newcommand{\diffusion}{\ensuremath{\ \text{Wm}^{-2} \text{K}^{-1}}}

\newcommand{\radians}{\ensuremath{^{\text{rad}}}}
\newcommand{\degrees}{\ensuremath{^{\circ}}}

%%%%% Document %%%%%
\begin{document}


\title{Orbital Constraints on Exoplanet Habitability}
\date{Submitted: \today{}}
\author{Zachery McBrearty}
\affiliation{\normalfont Level 4 Project, MPhys Physics with Astronomy\\ Supervisor: Dr Richard Wilman \\ Second Supervisor: Dr Craig Testrow \\ Department of Physics, Durham University}

\begin{abstract}

    A 1-D energy balance climate model is developed in order to investigate how changing certain orbital parameters can result in changes to a planet's habitability.
    Theoretical relationships between temperature, semimajoraxis, and eccentricity are derived from a simple 0-D energy balance model and are tested against the 1-D model and are found to be correct.
    A qualitative analysis of obliquity shows that there are optimal obliquities to minimise and maximise global temperature.
    The climates of exomoons orbiting gas giants are also investigated, including reflected light from the gas giant, eclipsing, and tidal heating.
    It is expected that these additional sources of heat move the habitable zones for the planet outwards.

\end{abstract}


\maketitle
%\thispagestyle{plain} % produces page number for front page

\tableofcontents
% \let\toc@pre\relax
% \let\toc@post\relax

\newpage

\section{Introduction} \label{sec:Introduction}
% Exoplanet/Exomoon habitability motivation 
% > Search for extraterrestrial life
% > Settling other worlds
% > Current missions to find exoplanets and exomoons


% Climate model
% > 0D-Equation and where it comes from
% > 1D-Equation + what the parameters are
% > Ignoring longitude dimension (How/Why)
% Earth as a baseline
% > using Earth to constrain certain parameters of the model to be realistic
% Exoplanets (i.e. moving Earth around)
% > Describe how varying parameters can lead to habitability parameter space
% > Suggest by generating parameter spaces that new discoveries can be quickly checked  
% Exomoons
% > How they differ from exoplanets
% > TH, Eclipsing, Reflectance
% > Depth model -> suggestion of life without requirements of light under an ice shell

\section{1-D Energy Balance Climate Model}\label{sec:1DEBCM}
% Rederive 1DEBCM from standard heat eqn
% Expand derivatives and introduce latitude as the standard parameter
\subsection{Characterising Model Parameters} \label{ssec:CharacterisingModelParameters} %%% -> could appendix if not enough space %%%
% Diffusion
% Heat capacity
% IR-emission
% Insolation -> why diurnally averaged (No longitude)
% Albedo -> snow / water reflectance
\subsection{Discretisation of the PDE} \label{ssec:DiscretisationPDE}
% choice of partial derivatives
% choice of edge case partial derivatives

\section{Earth-like model} \label{sec:EarthLikeModel}
% choice of 0.7-uniform
% choice of D_0 by fitting to NC97
% choice of spatial separation and time step sizes
% > relate to numerical stability appx 
% Show Earth: 
% > temperature distribution
% > habitability (human, bio) distribution
% > Climate cycles (eccentricity, obliquity)

\section{Exoplanets} \label{sec:Exoplanets}
\subsection{Investigating time-averaged solar flux} \label{ssec:InvTimeAveragedSolarFlux}
% use 0D model and Mendez to find parameter relations:
% > T prop a^-1/2
% > > talk about powerlaw being obeyed well until snowball
% > T prop (1-e^2)^-1/8
% > > good relationship until very high eccentricity
% > a prop (1-e^2)^-1/4
% > > finding inner and outer habitable zone (Human, Bio)
% % Describe in caption, analysis in main body of text 
\subsection{Investigating obliquity and rotation rate}\label{ssec:InvObliquityRotationRate}
% Qualitative look at how obliquity influences habitability
% > global temperature-obliquity
% > latitude-obliquity
% > time-obliquity (seen in Earth-like model?)
% Qualitative look at how rotation-rate influences habitability
% > "island of stability" at higher rotation rates
% > Three plot of obliquity showing the "island" growing / stabilising
\subsection{Investigating of ocean fraction} \label{ssec:InvOceanFraction}% -> lower priority
% Stability of the snowballing w.r.t. ocean fraction
% > a-f_ocean plot showing snowball transition changing?

\section{Exomoons} \label{sec:Exomoons}
\subsection{Investigating eclipsing} \label{ssec:InvEclipsing}
% Eclipsing amount vs a_gas, e_gas, a_moon, e_moon
% show that eclipsing is minor / could be approximated by powerlaws?
\subsection{Investigating tidal heating} \label{ssec:InvTidalHeating}
% with constant TH -> vary a_gas-e_gas to see how habitability moves
% const a_gas/e_gas -> vary a_moon-e_moon to see habitability relations
% vary planet size/radius/mass? 
\subsection{Developing and investigating a depth model} \label{ssec:DevInvDepthModel} % -> lower priority
% Derive depth model, referencing sec2
% setup system for ocean planet habitability under an ice layer with tidal heating

\section{Discussion} \label{sec:Discussion}
% Problems arising during analysis / method
% Sources of uncertainty in the method
% Approximations made and their validity
% Next steps in the work
% > most likely about the depth model and where it could be taken

\section{Conclusion} \label{sec:Conclusion}
% Summary
% > Why results matter
% > What the results are
% > Main problems with results
% > Main future goals of the research


% \begin{acknowledgments}
%     (OPTIONAL) The author would like to thank...
% \end{acknowledgments}

\bibliographystyle{ieeetr}
\bibliography{references}

\clearpage

\appendix

\section{Defining Equilibrium Temperature and Averages used} \label{appx:EquilTempAverages}
% Time-averaged temperature -> T(lat)
% Latitude-averaged temperature -> T(t)
% Time-and-latitude-averaged temperature -> <T>
% Equilibrium temperature as when <T_i+1> - <T_i> / <T_i> < epsilon

\section{Numerical stability of the 1D EBCM} \label{appx:NumStability}
% Maths analysis
% Varying spacedim to check analysis
% Varying timestep to check analysis

\section{Tidal heating equations and method} \label{appx:TidalHeatingEquationsMethod}
% What the equations are
% How they are combined

\clearpage

\section*{Scientific Summary for a General Audience}

Many interesting solar systems have been reported in the news, such as the Trappist-1 system which is filled with Earth-like planets.
Simulations and models such as those in this paper are used to determine if a planet could be habitable.
A habitable zone can be made by varying the parameters of the model to see where the model is habitable, partially habitable, or uninhabitable.

The main model in this paper takes a planet and divides it into a number of latitude bands which can have energy flow between them.
Certain parameters, such as how the planet orbits around it's star and the angle the planet is tilted at, are varied to build this habitable zone.
A result of this paper is if the Earth orbited slightly further away from the Sun then it is likely that it would fall into an ice age similar to what the Earth has experienced in the past.
Another result found is that the tilt of the planet can affect how hot or cold it is, and indicates that the current tilt of the Earth gives a cold planet.

Another aspect of this paper's exoplanet research is exomoons orbiting a gas giant such as Jupiter.
In certain configurations an exomoon can be heated not only from the host star, but also due to a process called tidal heating.
Tidal heating is similar to stretching an elastic band.
Stretching and relaxing an elastic band many times can cause the band to warm up.
The moon of a gas planet is stretched slightly by unequal forces of gravity as one part of the moon is further away than the other.
If the moon's orbit is not circular then the moon is stretched and relaxed, thus heats up in a similar way to the elastic band.
Adding tidal heating to the model allows for investigations into how tidal heating can move, or change the shape of, the habitable zone.

\end{document}